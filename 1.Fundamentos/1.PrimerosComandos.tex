%%%%%%%%%%%%%%%%%%%%%%%%%%%%% Comentarios %%%%%%%%%%%%%%%%%%%%%%%%%%%%%%%%%%
% Comentario de una línea %%%%%
% 1. El tipo de comentario más usado es Article
% 2. Los comandos comienzan con \
% 3. Todo documento inicia con documentclass
% Comentario de varias líneas %%%%%

%%%%%%%%%% Caracteres Especiales %%%%%%%%%%%%%%%%%
%%%  % para comentarios
%%%  ` comilla simple produce ` 
%%% pero `` produce " y para cerrar '' y obtenemos " "
%%%  # argumentos de entrada
%%%  & separador de tabulaciones
%%%  $ matemática en línea
%%% si necesitamos escribir cualquiera de los caracteres especiales
%%% anteriormente mencionadas debemos anteponer el \
%%% Tener muy en cuenta porque estos suelen romper 
%%% el documento
%%%%%%%%%%%%%%%%%%%%%%%%%%%%%%%%%%%%%%%%%%%%%%%%%%%%

%%% Espaciado: si utilizo muhcos espacios solo se reconoce como uno
%%% El número de Enters es lo mismo, solo usar uno
%%% Esta característica se denomina ifenación

%%% superíndices y subíndices
%%% Usamos ^ para superíndices
%%% _ para subíndices
%%% podemos usar {} para super-sub índices grandes

%%% Ambientes: Espacios de trabajo dentro de nuestro documento
%%% denotados por el begin y end
%%% Plantillas suelen tener dos tipos de docs: .sty (estilo) y .tex

\documentclass{article}
\usepackage[utf8]{inputenc} %para caracteres en español%
\usepackage[spanish]{babel} %para correcciones en español%
\begin{document}
    Primer texto:
    El principal grupo en ese momento correspondía al que en ese momento correspondía
    al que desde los años setenta fue ``bautizado'' como grupo Sudamérica, y que algunos
    llamaban el Sindicato Antioqueño y otros el Grupo Empresarial Antioqueño, con unos activos
    equivalentes al 15.7\% del PIB, unos \$11.500 millones de dólares estadounidenses cuando en
    los años setenta ocupaba el cuarto puesto, con activos equivalentes al 7.3\% del PIB; es decir
    más que duplicó su peso relativo.

    Diferencia del uso de matemática en línea:

    Sean tres enteros tales que c=a-b+1

    Sean tres enteros tales que $c=a-b+1$
    Manejo de índices:
    $a_n = a_{n - 1} + a_{n - 2}$

    $f(x) = e^{ax + b} - c$

    Letras Griegas:

    $P(z)=\alpha+\lambda(z - \alpha)^{n}$

    Integrales:

    $\int x$

    $\int x \,dx$

    $\int_a x \,dx$

    $\int_{H(s)} x \,dx$

    $\int_{a}^{b} x^{2x} \,dx $

    Derivada

    $f' f'' f'''$

    Poquito de funciones trigonométricas

    $\sen x + \ln y +  \cosh z$

    %%% Ejemplo de ambiente para itemsize%%%
    Ejemplo de lista de items
    \begin{itemize}
        \item Python
        \item Java
        \item R
        \item TF
        \item PyTorch
    \end{itemize}

\end{document}